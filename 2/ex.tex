\section{Vorbereitungsaufgaben}
\subsection{2.A}
Es gibt sehr viele Aufspaltungen der energetischen Niveaus durch die hohe Anzahl wechselwirkender Atome. Durch diese Delokalisierung der Ladung steigt die Möglichkeit und die diskreten Energieniveaus gehen in Energiebänder über.
\subsection{2.B}
In Halbleitern wäre ohne Dotierung kein Zustand im Leitungsband besetzt. Man benutzt Dotierung also, um freie Ladungsträger zu erhalten und die Leitfähigkeit zu verbessern. Man erreicht dies durch gezielte Verunreinigung (Atome anderer Stoffe werden in Kristallplätze gesetzt).
\subsection{2.C}
Sowohl Donatoren, als auch Akzeptoren sind Atome mit anderer Wertigkeit, als die ungebundenen Kristallatome. Sie bringen ein zusätzliches freies Elektron bzw. ein Loch in den Kristall und machen diesen dadurch einfach leitend.
\subsection{2.D}
Bestimmt wird die Schichtdicke durch $U_0$ (Diffusionsspannung), $N_D$/$N_A$ (Donator- bzw. Akzeptorkonzentration), $U$ (angelegte Spannung), $\epsilon_r$ (Permittivität).
\begin{eqnarray*}
\end{eqnarray*}