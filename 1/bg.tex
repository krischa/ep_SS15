\section{Theoretical background}
\subsection{Conducting properties}
If the electrical properties of a double-cable are equal on the whole cable, it is called homogeneous. In this experiment, we work with such cables.\\
Capacitive and inductive properties of the cable are:
\begin{eqnarray*}
C&=&\epsilon_r\epsilon_0l\frac{2\pi}{\ln\left(\frac{r_a}{r_i}\right)}\\
L&=&\mu_r\mu_0\frac{\ln\left(\frac{r_a}{r_i}\right)}{2\pi}
\end{eqnarray*}
The four characteristics of a cable\footnote{Resistance, inductance, capacity and loss} grow proportional to it's length. A lossless cable can be approximated as a chain of many LC-links.
\subsection{Expansion of waves in homogeneous cables}
\begin{eqnarray*}
	\frac{d^2}{dx^2}U-\gamma^2U&=&0\\
	\gamma^2&=&z'\cdot y'\Rightarrow\text{damping}\\
	solution: \quad U(x,t)&=&U_f(x,t)+U_b(x,t) \quad \text{f: forward, b: backwards}\\
	I(x,t)&=&I_f(x,t)+I_b(x,t)
\end{eqnarray*}