\section{Procedure}
\subsection{1.5.1 Differentiator}
The oscilloscope is triggered external and the rectangular signal is differentiated by a high-pass filter.\\
We observe, what happens if we use the RC-link with a build in Resistance of $\SI{2.2}{\kilo\ohm}$.
\subsection{1.5.2 Pulses in cables}
To understand how pulses spread in cables with both ends open, we observe the voltage of pulses send through a $\SI{50}{\ohm}$-cable at the beginning and the middle of the cable. The input resistance is very large compared to the $\SI{50}{\ohm}$ which allows the pulses to get into the cable but also allows us to approximate the cable terminal open.
\subsection{1.5.3 Cable termination, delay}
\subsubsection{a)/b)}
The delay cable terminal is open
\subsubsection{c)}
The delay cable terminal is short-circuited.
\subsubsection{d)}
Varying frequencies.
\subsection{1.5.4 Clipcable, damping}
Often long pulses shall be shortened to a defined length. To do so we use clip cables, which use reflexion to shorten the pulses.
\subsubsection{a)}
Open terminal.
\subsubsection{b)}
Short-circuited clip cable.
\subsubsection{c)/d)}
We vary the frequencies and observe the distances between pulses. Then we use a $\SI{2}{\meter}$ clip cable and consider on what thepulse length depends.
\subsubsection{e)}
 The specific damping of the HH 2500 shall be determined. To do so, we must measure the ratio of the top-stage to the stage one step below.
\subsection{1.5.5 $\SI{50}{\ohm}$-Cable RG-58 C/U}
In most cases we want the cable to damp or distort incoming pulses as little as possible. In reality this is for different cables only possible for certain bandwidths.\\
Modern transmission cables are often used with a large bandwidth and have little wave resistance ($\approx\SI{50}{\ohm}$) and also little delays.