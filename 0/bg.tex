\section{Theoretical background}
…\subsection{Amplitudes of AC voltage}
To describe AC voltage amplitudes, the following voltages can be used:
\begin{eqnarray*}
	\text{peak-to-peak-voltage:}\quad U_{PP}&=&max\{U(t)\}-min\{U(t)\}\\
	\text{peak-voltage \tiny{(symmetric and most times also asymmetric voltage)}:}\quad U_P&=&max\{|U(t)|\}\\
	\text{root-mean-square-amplitude-voltage:}\quad U_{RMS}&=&\sqrt{\left<U^2(t)\right>}
\end{eqnarray*}
\subsection{Measuring instruments}
Every measuring instrument is made up of two parts:
\begin{itemize}
	\item{Measuring unit}
	\item{Displaying unit}
\end{itemize}
\subsubsection{Measuring unit}
The measuring unit measures the measurand and controls the displaying unit. It is made up itself of another 3 subunits:
\begin{itemize}
	\item{Measuring amplifier}
	\item{Area selection network}
	\item{Measuring converter}
\end{itemize}
\subsubsection{Displaying unit}
The displaying unit allows us to read the value of the measurement. If the measuring instrument is an analogue one, it consists of a pointer a coil which steers the pointer and a scale. If it is a digital one, it consists of a counter which -most times- gets it's input from a so called dual-slope converter\footnote{it measures a measurand by the time it takes to decharge a build-in capacitor} and a LCD.
\subsection{The oscilloscope}

